%%%%%%%%%%%%%%%%%%%%%%%%%%%%%%%%%%%%%%%%%%%%%%%%
%
% == AIT CSIM Handout LaTeX Template ==
% == Credit ==
% Assoc. Prof. Matthew N. Dailey
% Computer Science and Information Management
% Asian Insitute of Technology
%
%%%%%%%%%%%%%%%%%%%%%%%%%%%%%%%%%%%%%%%%%%%%%%%%

\documentclass{article}

\usepackage{a4,url,upquote}
\usepackage{graphicx}
\usepackage{hyperref}
\usepackage[cmex10]{amsmath}
\usepackage{amssymb}
\usepackage{placeins}

\setlength{\textwidth}{6.5in}
\setlength{\textheight}{9in}
\setlength{\oddsidemargin}{0in}
\setlength{\evensidemargin}{0in}
\setlength{\topmargin}{0in}
\setlength{\headheight}{0in}
\setlength{\headsep}{0in}
\setlength{\footskip}{0.5in}

\newcommand{\bheading}[1]{\vspace{10pt} \noindent \textbf{#1}}

\begin{document}

\begin{tabbing}
    \`\=\kill
    \textbf{Workshop:} The WordPress Way
    \` September 11, 2015 \\
    \textbf{Instructor:} Kan Ouivirach ({\tt \small kan@prontomarketing.com})
        and Navarant Pramuksan ({\tt \small oy@prontomarketing.com}) \\
    \textbf{Company:} Pronto Marketing (Research and Develpment Team)
\end{tabbing}

\hrule

\vspace{.25in}

\begin{center}
    \textbf{\Large The WordPress Way: Customizing WordPress Theme and
        Developing WordPress Plugins}
\end{center}

\vspace{.15in}

\noindent \textbf{Introduction:} This workshop is intended for anyone who wants
    to learn WordPress and do things the WordPress way. Participants will
    learn how to customize a WordPress theme and better understand how
    WordPress template hierarchy works. Participants will also learn
    how to develop a WordPress plugin to enhance the funtionality of the theme.
    By the end of this workshop, participants will be able to develop their own
    WordPress theme and use WordPress as a solution to real-world problems.

\section*{Customizing WordPress Theme}

\noindent In this section, we will customize a WordPress theme, namely
    ThirtyFifteen, and follow the WordPress template hierarchy~[1] to extend
    a new template for a new custom post type. \\

workshop. Figure~\ref{fig:csim-logo} shows
a logo of CSIM. \\

\begin{itemize}
    \item[-] {\tt code} - Folder that contains code and scripts
    \begin{itemize}
        \item[-] File {\tt workshop1.py}
        \item[-] File {\tt workshop2.py}
    \end{itemize}
    \item[-] {\tt data} -- Folder that contains the datasets.
    \begin{itemize}
        \item[-] {\tt input} -- Folder that contains the input images.
        \begin{itemize}
            \item[-] {\tt folder\_1} -- Folder that contains some files.
            \item[-] {\tt folder\_2} -- Folder that contains some other files.
        \end{itemize}
    \end{itemize}
\end{itemize}

\FloatBarrier

\section*{Developing WordPress Plugin}

\noindent We will do a simple workshop. Figure~\ref{fig:csim-logo} shows
a logo of CSIM. \\

\begin{figure}[t]
    \centering
    %\includegraphics[width=2in]{figures/csim}
    \caption{CSIM Logo}
    \label{fig:csim-logo}
\end{figure}

\noindent Let's follow these steps below.

\begin{enumerate}
    \item Step 1
    \item Step 2
    \item Step 3
\end{enumerate}

\section*{References}

\begin{itemize}
    \item[1] WordPress Template Hierarchy -- \tt{http://wphierarchy.com/}
\end{itemize}

\end{document}
